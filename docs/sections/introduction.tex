\section{Introduction}
The camera pose can be expressed through two component:
\begin{enumerate}
    \item a tuple of three elements that identifies the coordinates $x,y\text{ and }z$
    \begin{equation}
        x_c=(x,y,z)\quad x,y,z \in \mathbb{R}  
        \label{eq:absolute-position-definition}
    \end{equation}
    \item a quaternion of four elements that identifies the rotation of the camera
    \begin{equation}
        q_c=(qw, qx, qy, qz)\quad qw,qx,qy,qz \in \mathbb{R}
        \label{eq:quaternion-as-rotation-definition}
    \end{equation}

\end{enumerate}
Consequentially the pose is referred as $p_c=(x_c, q_c)$. It is important to notice that this is not the only available representation of a pose.

Given an image $I_c$ captured by a camera $C$, an absolute pose estimator $E$ tries to predict the $3D$ pose orientation and location of $C$ in world coordinates, defined for some arbitrary reference $3D$ model. The \textit{absolute pose estimation (APE)} problem can be formally defined as the problem of estimating a function $E$ taking an image $I_c$ caputered by a camera $C$ and outputting its respective pose:
\begin{equation}
    E(I_c) = (x_c, q_c)
    \label{eq:absolute-pose-estimation-task}
\end{equation}

Another problem related to APE is \textit{relative pose estimation (RPE)}, in this kind of task the estimator takes two image $I_c^1$ and $I_c^2$ captured by $C$ and aims to predict the relative pose between them. The equation \ref{eq:absolute-pose-estimation-task} becomes:

\begin{equation}
    E(I_c^1, I_c^2) = (x_c^{rel^2}, q_c^{rel})
    \label{eq:relative-pose-estimation-task}
\end{equation}

where $x_c^{rel}$ can be the absolute pose with \textit{coordinates reference system} in $I_c^1$ or a translation vector from $I_c^1$ to $I_C^2$.
