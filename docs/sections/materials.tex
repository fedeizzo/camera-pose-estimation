\section{Materials}
Every material used in the project have been uploaded respectively:
\begin{itemize}
    \item the datasets have been uploaded on the Google Drive folder;
    \item the code is available in the GitHub repository.
\end{itemize}

The project has been developed in Python 3, using common data science libraries, such as numpy, pandas, PyTorch, matplotlib, scipy, and many others.

\subsection{Repository organization}
The repository follows the structure:
\begin{itemize}
    \item \texttt{camera-pose-estimation/}
    \begin{itemize}
        \item \texttt{model/} contains everything related to the deep learning part of the project. It also includes the code used for implementing the web server under \texttt{webserver.py} and \texttt{static/}.
        \item \texttt{tools/} contains scripts used for the dataset generation pipeline.
    \end{itemize}
    \item \texttt{config\_parser/}: Python package written by us that allows to create configuration files, with the idea of improving reproducibility in our experiments. Each configuration file can be subdivided in sections: for each section you can define variables with the sintax \texttt{label=value}, where \texttt{value} is a parsable JSON object (boolean, int, float, list, object).
    \item \texttt{notebooks/} contains some Python Jupyter Notebooks that have been used for data exploration, validation, and post-processing of the model predictions.
\end{itemize}

\subsection{Data organization}
For each footage, a folder has been created:
\begin{itemize}
    \item \texttt{imgs/} contains the video frames exported with ffmpeg;
    \item \texttt{processed\_dataset/} contains the train, validation, and test datasets that can be reused during different trainings: this helps speeding up the loading procedure from \dots minutes to \dots seconds;
    \item \texttt{workspace/} contains the models generated by COLMAP;
    \item each of \texttt{train.csv}, \texttt{validation.csv}, and \texttt{test.csv} contains a table for specifying the pose for each image frame. This are the files generated with the \texttt{video\_to\_dataset.sh} script.
\end{itemize}
